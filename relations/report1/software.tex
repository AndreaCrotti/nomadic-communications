\subsection{Software} \label{setup:software}
	% ssh keys, iwconfig, python software, iperf, tcpdump, gnuplot, whatever used


%For setting up our network we used the iwconfig

\noindent
For measuring the traffic we wrote a python script which makes use of other software.

Python is a simple, interpreted, object oriented scripting language with a large set of libraries.
These are the softwares we used for our project:

\begin{description}
	\item[iperf]
		measures throughput of traffic between two hosts, not limited to wireless setups.
It has a long list of possible options that can be used to test the network with many different low level settings.

The tool can do UDP and TCP traffic tests.\newline
While TCP would measure the throughput which the end-user experiences in most cases, the connectionless UDP option is closer to the low-level details which we were interested in.
To test the network with iperf we need to launch a daemon one host and connect to it.\newline
If we set the interval flag on both the server and the client we get the resulting bandwidht from both, but as we'll see later they could differ.\newline
In fact using the udp mode the client doesn't actually know anything of what is the server actually receiving.
That's why we decided to collect both outputs for every test done.

The only way to understand that the client's results are not precise is to look at the miss rate.
		% more detail?

	\item[tcpdump]
		is a packet sniffer which logs packets sent through one interface. Optionally you can specify a filter on the command line to get only packets you are interested in.
The output file is a binary format, ({\em pcap}), which can be then read by many other applications like Wireshark.
		
	\item[Wireshark]
		is a graphical tool which helps you analyze captured traffic by parsing packets and displaying structured information about them.
		We used this tool in the beginning to see where we can refine our work, but in the end the program was not used anymore.

	\item[gnuplot]
		is an application which can represent data in graphs of many types. It has an interactive text interface, as well as it can be used as a library by a piece of software. We used it with the help of the wrapper library ``python-gnuplot'' for python.

	\item[paramiko]
		is a python library implementing the SSH2 protocol.	
	This is used to enable the python script which automates the tests to launch automatically the iperf server and the tcpdump monitor.

\end{description}
%
The python script was mainly written by Andrea Crotti.  Its main goal is to
automate as many tasks as possible to do our measurement.  The main strength
of the program is its configurability.  From simple configuration ({\em ini})
files it reads the required information about the network topology as well as
the settings to apply when testing the throughput.

A comprehensive description can be found in the file {\tt README.textile}.


\subsubsection{Test automator script}

\noindent
The script \verb|tester.py| is the entry point for the program, and it has support for per-user configuration:
To invoke a test, the program is launched as {\tt ./tester.py \textit{name}},
after placing a configuration file at the location {\tt userconfs/\textit{name}.ini}.\newline

There is also a global configuration file, {\tt default.ini}, which is read at first.
Its values have a lower precedence, thus they are overridden by the per-user configuration file.

For different tests, there are again configuration files inside the {\tt configs} directory, which in turn override some of the configuration before doing the actual experiment.

By default all the configurations get loaded, so every test is a three-merge of default + user\_defined + test\_defined.\newline
This makes a lot easier to define new test sets or test same parameters on different networks.

The contents of these files are described in the next section.


%For the options we considered, see Section \ref{sec:parameters}.

