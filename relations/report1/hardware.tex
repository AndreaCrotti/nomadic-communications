\subsection{Hardware} \label{setup:hardware}
	% mainly C wifi card configuration, ap settings

\noindent
For the experiment we were provided with some devices by the university, however, we decided to use our own equipment for doing the tests.

Our iperf server and client were run on different notebooks, as shown in table \ref{tbl:laptops}. Furthermore, the network consists of an US Robotics Access Point and a switch from the same company (table \ref{tbl:networkdevices}).


%\subsubsection{Laptops}
%	\begin{tabular*}{1\textwidth}{@{\extracolsep{\fill}} | l l | }
	\begin{table}[h]
		
		\begin{tabularx}{15cm}{ | m{4cm} X | }
			\hline
				ROLE & Client\\
				MODEL & HP Pavilion dv5-1070el\\
				WIFI & Intel PRO/Wireless 5100 AGN\\
				FIRMWERE & iwlwifi-5000 - 6 October 2008\\
			\hline
		\end{tabularx}
		\\\\\\
		\begin{tabularx}{15cm}{ | m{4cm} X | }
			\hline
				ROLE & Server\\
				MODEL & Asus M2410L\\
				WIFI & Intel PRO/Wireless LAN 2100 3B Mini PCI\\
				FIRMWERE & ipw2100\\
			\hline
		\end{tabularx}
		
		\caption{Laptop configuration}
		\label{table1}
		\label{tbl:laptops}
	\end{table}

%\subsubsection{Network devices}
	\begin{table}[h]
		
		\begin{tabularx}{15cm}{ | m{4cm} X | }
			\hline
				ROLE & Access Point\\
				MODEL & US Robotics 805450\\
				FIRMWERE & version 1.53\\
			\hline
		\end{tabularx}
		\\\\\\
		\begin{tabularx}{15cm}{ | m{4cm} X | }
			\hline
				ROLE & Switch\\
				MODEL & US Robotics\\
			\hline
		\end{tabularx}
		
		\caption{Network devices configuration}
		\label{table2}
		\label{tbl:networkdevices}
	\end{table}
