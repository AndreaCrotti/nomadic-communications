% Everything that comes from theory and that is needed to understand and describe our work.
	\subsection{802.11} \label{theory:prot_specs}
	802.11 is the standard family of protocols for wireless networks. It is also named Wi-Fi. 802.11 protocols family describe how various wireless devices must interact. Nowadays the most popular 802.11 implementation are 802.11b and 802.11g.
	
% A simple description of how an IEEE 802.11 protocol works to introduce the fact that there is some overhead added by the protocol.
% Yes, I know it's unbelivable but there are some differences, for real! :)
	\subsection{Differences between 802.11b and 802.11g} \label{theory:prot_differences}
	802.11b protocol was born in October 1999 and since this date it has a discrete success but with the advent of some other wireless device such as cordless telephones it suffer interfernce that couses slower performance.\\
	In June 2003 arrives 802.11g that uses the same basic idea of 802.11b, it's fully backwards compatible but has best bit rates and throughput\\
	Both protocols use the same frequency band (2.4 GHz) but use different frequency spreads. While 802.11b use direct sequence spread spectrum signaling (DSSS), 802.11g use orthogonal frequency division multiplexing (OFDM) methods.
	
		
		% Calculate protocols upperbound limitations and similar things (report the formulas!).
		% DO NOT USE PROFESSOR SLIDES AS SOURCE OF INFORMATIONS SINCE MOST OF VALUES ARE COMPLETELY WRONG!!!!!!!!!
		% Download the IEEE standard from here: http://disi.unitn.it/locigno/didattica/NC/802.11-2007.pdf and make a large use of ^F.
