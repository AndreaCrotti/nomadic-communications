\subsection{Software} \label{setup:software}
	% ssh keys, iwconfig, python software, iperf, tcpdump, gnuplot, whatever used


%For setting up our network we used the iwconfig

\noindent
For measuring the traffic we wrote a python script which makes use of other software.

Python is a simple, interpreted, object oriented scripting language with a large set of libraries.
These are the pieces of software we used for our project:

\begin{description}
	\item[iperf]
		measures throughput of traffic between two hosts, not limited to wireless setups.
		It only has a command-line interface, with many options to configure its behavior, some of which we changed throughout our tests, while keeping others the same.

		The tool can do UDP and TCP traffic tests.
		While TCP would measure the throughput which the end-user experiences in most cases, the connectionless UDP option is closer to the low-level details which we were interested in. We chose to only use this feature.

		TODO more detail?

	\item[tcpdump]
		is a packet sniffer which logs packets sent through one interface. Optionally you can specify a filter on the command line to get only packets you are interested in.
		The output file is a binary format, called {\tt pcap}, which can be read by many other applications like Wireshark.
		
	\item[Wireshark]
		Wireshark is a graphical tool which helps you analyze captured traffic by parsing packets and displaying structured information about them.
		We used this tool in the beginning to see where we can refine our work, but in the end the program was not used anymore.

	\item[gnuplot]
		is an application which can represent data in graphs of many types. It has an interactive text interface, as well as it can be used as a library by a piece of software. We used it with the help of the wrapper library for python ``python-gnuplot''.

\end{description}
